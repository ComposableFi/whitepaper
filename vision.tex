\section{Vision}

The decentralized finance (DeFi) ecosystem has sought sharded blockchains for increased scalability. As examples, ETH 2.0, Polkadot, and NEAR will be sharded. The result, however, is that even though the vision of ETH 2.0 is already upon us, instead of just blockchains being sharded, applications are being sharded as well. As an example, SushiSwap is deployed on Ethereum Virtual Machine (EVM) chains, Layer 2s (L2s) like rollups, Parachains, etc., and the expansion of these applications to other ecosystems is very likely. Thus, while moving money intra-ecosystem is becoming more intuitive, with several applications segregated within a specific ecosystem, managing assets inter-ecosystem is not.

Thus, Composable is focused on a cross-chain, cross-layer liquidity layer for sharded applications. Composable takes this notion a step further by also realizing that between each ecosystem, there is a sharding of functionality itself; for instance, every ecosystem has its own lending protocol. Composable’s vision is thus to abstract away inter-ecosystem decision making and maximize users’ and developers’ outcomes based on their unique goals.

To accomplish this function, we are creating an inter-ecosystem communication protocol, utilizing a parachain as a finality layer, that will connect L2s, L2s to the Polkadot/Kusama ecosystem, and the Cosmos Ecosystem through the Inter-Blockchain Communication protocol (IBC) to the Polkadot/Kusama ecosystem. The hope is then we would expand this to include other ecosystems, such as Algorand, Solana, etc. With these connections, an action such as borrowing X against Y would be routed through an ecosystem-agnostic path, maximizing for users’ preferred outcomes and parameters. The applications of such a stack are limitless.
