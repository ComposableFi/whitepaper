\section{Cross-Chain Virtual Machine\label{sec:xcvm}}
The cross-chain virtual machine (XCVM) is a single, developer friendly interface to interact and orchestrate smart contract functions across the multitude of L1 and L2 networks available. In short, the XCVM abstracts complexity from the process of having to send instructions to the routing layer directly, initiates call-backs into smart contracts, and handles circuit failure such as network outages.

Build on top of our bridging infrastructure, our tools for the Composable Cross-Chain Virtual Machine allow developers to tap into various functions of communication and liquidity availability. The result is multifaceted; users can perform cross-chain actions, and the overarching blockchain ecosystem is repositioned as a network of agnostic liquidity and available yield.

Composable allows developers to to tailor their experience to maximize for a desired parameter while minimizing ecosystem-specific decision making.

\subsection{Architecture\label{sec:xcvmarch}}
The virtual machine is, first and foremost, and abstract definition of how services should assume that cross-layer transfers and functionality operates, including security, finality, fee model and availability. The security and finality are dependent on the bridging technology used, and relatively static (although we will see that the security for optimistic bridges is a function of network participants and economic stake). Fees are highly dynamic, dependent on network traffic. Availability describes the existence of appropriate relayers, and even the existence of the destination chain. Although many systems are not designed to take into account deprecation of a blockchain, any truly resilient and scalable bridging technology must handle that failure mode.

We distinguish between two categories of bridges: optimistic and final. Mosaic is an example of an optimistic bridging technology, using decentralized relayers, economic incentives and dispute resolution to secure a network. IBC and XCM are both final, and require that the destination chain has a light client embedded, as well as deterministic finality. Tendermint, GRANDPA and BEEFY provide deterministic finality, while proof of work based chains can only be bridged using optimistic solutions, although these solutions are highly secure in the case of Bitcoin.

Smart-contracts usually assume two modes of result, success or failure, but are ill-equipped to handle partial success. Even relatively simple layer-to-layer operations may reach an indefinite state, where contracts state or funds are left in limbo. Instead of massively rewriting existing smart-contracts and protocols to handle the multitude of failure modes, the XCVM transparently handles state transitions, disputes and reverts. It combines different bridging protocols, such as IBC, XCMP and Mosaic Phase 3, and is capable of integrating new bridging technologies. Rather than modeling a cross-chain transfer(XCT) as a single operation, such as locking funds in a local contract, we model it as a set of reversible state transitions with different approximate costs for each transition. The VM exposes the lower level APIs to directly query and interact with the current state of the , as well as a higher level interface, that observes the complete set as a single, fallible operation. Sets may be combined into larger operations, allowing to roll-back the entire transaction.

By modeling XCTs as a state machine, or more accurately a directed, cyclic, weighted graph, we naturally arrive at the actual requirements of the XCVM: any blockchain that is capable of executing a transition within the XCT is XCVM compatible. Completely trustless compatibility can be achieved through the use of smart-contracts and runtime-embedded light clients, such as IBC. For chains without smart-contract capabilities or light clients, such as Bitcoin, we can employ optimistic bridging and message passing technologies, such as our Mosaic project. 

The XCVM offers Picasso based dapps different hooks and updates on the status of any XCT, as well as RBAC based flow control for actively managing the execution of different stages. Dapps incur transaction fees for calling into this underlying execution layer. Depending on the stage of the XCT, this fee may be subtracted from the XCTs payload in case of a simple token transfer using our bring-your-own-gas features.

\subsection{Well-Known Protocol Types}
Optimistic bridges are secured through a dispute resolution system, where relayers use a stake to provide for settlements in case of fraud. Properly securing an XCT requires knowledge of the value of the XCT. For token transfers, a pricing oracle may provide an estimate of the absolute stake required for a safe collaterization ratio. We can extend this security model to decentralized finance products by providing on-chain models of common protocols. Oracles then provide time-weighted prices and reserves, and locally we can compute the necessary collaterization to secure a sequence of DeFi operations within an XCT. Using well-known protocol types (WKPT), we can also define how to reverse an XCT, as well as locally check the validity.

WKTPs are identified by their contract hash (for smart-contract based protocols), Call definition (for XCM compatible chains) or module version (for IBC chains). Only WKTPs used by a specific XCT incur a runtime cost. The storage cost per WKTP ranges from a few kilo bytes to hundreds, meaning that the protocol can scale to millions of WKTPs on modern hardware. Depending on different factors, WKTPs could be effectively scaled using zero-knowledge technologies, requiring only the identifier and oracle data to be stored on-chain. 

Composable dapp developers may choose to create XCTs tied to specific identifiers, or a more general, interface based approach. The identifier approach has better security guarantees, but loses flexibility in the presence of upgradable contracts. This distinction is necessary, as an upgradable contract is able to arbitrarily change semantics, while binding the XCT to a specific contract hash. Not all upgradable contracts can be uniquely identified, as keeping track of a chain of logic delegations within upgradable contracts is prohibitively expensive. In that case XCT authors may only use the interface based approach.