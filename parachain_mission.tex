\section{Parachain Mission}

As mentioned previously, we will be pursuing the operation of both a Kusama and Polkadot parachain.

Given that our parachains are the foundational layer that powers our ecosystem, we have adopted a pallet-centric approach to adding products on our parachains. Meaning, we will offer projects the ability to deploy as pallets on our chain, with governance having the ability to upgrade these pallets into the runtime of our chains. We are excited to be able to offer this to the Kusama ecosystem, and have a grants programs for others to develop pallet projects using our technology, to be implemented into our parachain.

Projects that do well on the Kusama chain, can then upgrade to our Polkadot parachain.

Polkadot offers plug-and-play security, allowing Composable to focus on building its ecosystem, and leaving the security to Polkadot’s validators.
%
In other ecosystems, we would have had to recruit our own validators for security, which is a lot of work. We also chose Polkadot for its blockchain development framework, Substrate. Substrate allowed us to custom-build our blockchain, allowing us to continuously upgrade our blockchain with new functionalities without needing to fork the network.
%
Polkadot also allows for native cross-chain communication with all other parachains connected to Polkadot, as well as all external networks bridged to Polkadot.
%
Last, we believe Polkadot has the top engineering team and leadership in the industry, having been built by Gavin Wood who coded Ethereum, invented the Solidity programming language, and invented the Ethereum Virtual Machine (EVM). We believe Polkadot is building the third phase of crypto after Bitcoin and Ethereum.
