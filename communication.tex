\section{Finality and Application Layer}
Composable’s vision is to create a protocol that allows for communication across ecosystems. The result is a Port Control Protocol like system for blockchains. The end result is multifaceted; users can perform cross-chain actions, and the overarching blockchain ecosystem is re-positioned as a network of agnostic liquidity and available yield. Throughout these interactions, Composable allows users to tailor their experience to maximize for a desired scalar, such as security or speed, while minimizing ecosystem-specific decision making.

Instead of combining bridging infrastructure to our parachain's consensus, we consistently choose to keep these entities separate, ensuring that we are capable of upgrading and deprecating protocols with minimal impact on other infrastructure through our Innovation Availability Layer (IAL). 

\subsection{Finality}
Composable's support of different ecosystems implies that we must be able to control and abstract for the final user, the inclusion of transactions in different chains. Inclusion is tightly related to the concept of finality. Finality guarantees that past events on the blockchain are immutable, therefore when a transaction it is included on a final block we can be sure that it has been included on the chain. Unfortunately, strong finality cannot be provided without some compromises \cite{Brewer2012Cap} and most blockchains only offer some degree of finality. We list the three degrees of finality most likely to be found on different networks, from weaker to stronger finality:

\begin{itemize}
    \item \textbf{Probabilistic finality:} Finality is reached eventually. Under some assumptions, we can estimate the probability that a given block is considered final. With each new block added to the chain, older blocks become more final. E.g: Bitcoin and most PoW chains consider a block final after 6 blocks since the \textit{probability} of a fork decreases exponentially as the chain grows.
    
    \item \textbf{Provable finality:} In an effort to provide stronger and faster finality, some chains include some kind of finality gadget that runs in parallel to the  chain and performs come Byzantine agreement process over the blocks. Once the gadget has gone over those blocks and a consensus is reached, they are considered final. E.g: \href{https://github.com/w3f/consensus/blob/master/pdf/grandpa.pdf}{GRANDPA} on Polkadot and \href{https://arxiv.org/pdf/1710.09437.pdf}{Casper FFG} on Ethereum.
    
    \item \textbf{Absolute finality:} At a cost, some blockchains implement Probabilistic Byzantine Fault Tolerant (PBFT) consensus protocols. This means, once the block is crafted, it is automatically considered final (e.g: Tendermint).
\end{itemize}

We need to handle chains with different finalities and different synchronization times. Therefore, we cannot proceed with a deterministic solution. In order to make Composable decentralized and protocol-agnostic, we need to rely on validators.

\subsection{Innovation Availability Layer}
Our core design is centered around integration of cutting edge technologies across ecosystems, such as IBC, XCMP and zero-knowledge based bridges. The Innovation Availability Layer offers a protocol to upgrade bridging technologies without downtime, as well as an API stability guarantee for smart-contracts. Users and protocols alike will be able to interact through this singular interface to dispatch complex operations.

Foundational to our approach of expanding on existing, cutting-edge technology is our effort on stabilizing BEEFY and leading the charge on the Cosmos-Substrate bridging infrastructure. We are developing the reference implementation for BEEFY in Golang, as well as the BEEFY-IBC client libraries needed to support connections to Cosmos chains.
 
\subsection{Parachain}
At the core of our communication stack lies the parachain, functioning as a finality layer for IBC compatible chains, as well as a gateway into XCM compatible chains. It functions as the incentivization layer for light client data storage and proving. Polkadot and Kusama also allow for native cross-chain communication with all other parachains connected to the relay chain, as well as all external networks. An inflationary reward mechanism is used to incentivize collators and oracles. Through a runtime upgrade enacted by decentralized governance, the actual reward rate can be set and reduced as the protocol starts generating significant fees. Users and infrastructure providers are capable of staking PICA and LAYR tokens to grow with the ecosystem, providing critical security and capabilities to our ecosystem. 

\subsection{Pallets}
Given that our parachains are the foundational layer that powers our ecosystem, we have adopted a pallet-centric approach to adding products on our parachains. Meaning, we will offer projects the ability to deploy as pallets on our chain, with decentralized, stake-based governance, having the ability to upgrade these pallets into the runtime of our chains. We are excited to be able to offer this to the Kusama ecosystem, and have a grants programs for others to develop pallet projects using our technology, to be implemented into our parachain. Projects that do well on the Kusama chain, can then upgrade to our Polkadot parachain, which we envision to be the stable, more mature sister of our Kusama chain. 

Although we intent for untrusted sources to also perform protocol-to-protocol interactions on our parachains, through the web assembly-based XCVM, we initially focus on projects deploying as pallets, which allows these projects more granular, lower level access to our cross-chain APIs, as well as more advanced logic related to the block life-cycle, storage and cryptographic primitives. We are strong believers in security through tooling and support the move within the DeFi ecosystem towards Rust and web assembly as tools and computing platforms.